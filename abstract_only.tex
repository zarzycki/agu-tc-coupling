%% Copernicus Publications Manuscript Preparation Template for LaTeX Submissions
%% ---------------------------------
%% This template should be used for copernicus.cls
%% The class file and some style files are bundled in the Copernicus Latex Package which can be downloaded from the different journal webpages.
%% For further assistance please contact the Copernicus Publications at: publications@copernicus.org
%% http://publications.copernicus.org


%% Please use the following documentclass and Journal Abbreviations for Discussion Papers and Final Revised Papers.


%% 2-Column Papers and Discussion Papers
\documentclass[gmd, manuscript]{copernicus}

%% Journal Abbreviations (Please use the same for Discussion Papers and Final Revised Papers)

% Archives Animal Breeding (aab)
% Atmospheric Chemistry and Physics (acp)
% Advances in Geosciences (adgeo)
% Advances in Statistical Climatology, Meteorology and Oceanography (ascmo)
% Annales Geophysicae (angeo)
% ASTRA Proceedings (ap)
% Atmospheric Measurement Techniques (amt)
% Advances in Radio Science (ars)
% Advances in Science and Research (asr)
% Biogeosciences (bg)
% Climate of the Past (cp)
% Drinking Water Engineering and Science (dwes)
% Earth System Dynamics (esd)
% Earth Surface Dynamics (esurf)
% Earth System Science Data (essd)
% Fossil Record (fr)
% Geographica Helvetica (gh)
% Geoscientific Instrumentation, Methods and Data Systems (gi)
% Geoscientific Model Development (gmd)
% Geothermal Energy Science (gtes)
% Hydrology and Earth System Sciences (hess)
% History of Geo- and Space Sciences (hgss)
% Journal of Sensors and Sensor Systems (jsss)
% Mechanical Sciences (ms)
% Natural Hazards and Earth System Sciences (nhess)
% Nonlinear Processes in Geophysics (npg)
% Ocean Science (os)
% Proceedings of the International Association of Hydrological Sciences (piahs)
% Primate Biology (pb)
% Scientific Drilling (sd)
% SOIL (soil)
% Solid Earth (se)
% The Cryosphere (tc)
% Web Ecology (we)



%% \usepackage commands included in the copernicus.cls:
%\usepackage[german, english]{babel}
%\usepackage{tabularx}
%\usepackage{cancel}
%\usepackage{multirow}
%\usepackage{supertabular}
%\usepackage{algorithmic}
%\usepackage{algorithm}
%\usepackage{float}
%\usepackage{subfig}
%\usepackage{rotating}

% cmz - Additional added packages
%\usepackage{rotating}
%\usepackage{color}


\begin{document}

\nolinenumbers
%\linenumbers

\title{Impact of ocean coupling strategy on extremes in high-resolution atmospheric simulations}


% \Author[affil]{given_name}{surname}

\Author[1]{Colin M.}{Zarzycki}
\Author[2]{Kevin A.}{Reed}
\Author[1]{Julio}{Bacmeister}
\Author[1]{Anthony P.}{Craig}
\Author[1]{Susan C.}{Bates}
\Author[1]{Nan A.}{Rosenbloom}

\affil[1]{Climate and Global Dynamics, National Center for Atmospheric Research, Boulder, Colorado, USA.}
\affil[2]{School of Marine and Atmospheric Sciences, State University of New York at Stony Brook, Stony Brook, New York, USA.}

%% The [] brackets identify the author with the corresponding affiliation. 1, 2, 3, etc. should be inserted.

\runningtitle{OCEAN COUPLING IMPACT ON CLIMATE EXTREMES}

\runningauthor{ZARZYCKI ET AL.}

%\correspondence{Colin M. Zarzycki, Climate and Global Dynamics, National Center for Atmospheric Research, Boulder, Colorado, USA. (zarzycki@ucar.edu)}

%\received{}
%\pubdiscuss{} %% only important for two-stage journals
%\revised{}
%\accepted{}
%\published{}

%% These dates will be inserted by Copernicus Publications during the typesetting process.

\firstpage{1}

\maketitle

\begin{abstract}

This paper discusses the sensitivity of tropical cyclone climatology to ocean coupling strategy in high-resolution configurations of the Community Earth System Model. Using two supported model setups, we demonstrate that the choice of grid on which the lowest model level wind stress and surface fluxes are computed may lead to differences in cyclone strength in multi-decadal climate simulations, particularly for the most intense cyclones. Using a deterministic framework, we show that when these surface quantities are calculated on an ocean grid that is coarser than the atmosphere, the computed frictional stress is misaligned with wind vectors in individual atmospheric grid cells. This reduces the effective surface drag, and results in more intense cyclones when compared to a model configuration where the ocean and atmosphere are of equivalent resolution. Our results demonstrate that the choice of computation grid for atmosphere/ocean interactions is non-negligible when considering climate extremes at high horizontal resolution, especially when model components are on highly disparate grids.

\end{abstract}

\end{document}
